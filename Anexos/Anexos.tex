\section{Connection on a vector bundle}


Given a smooth function $f=(f^{1},\dots,f^{m}):M\rightarrow\mathbb{R}^{m}$ and
a vector field $X\in\mathfrak{X}(M)$ it makes sense to consider the derivative
of $f$ in the direction of $X$:
\[
X(f)=(X(f^{1}),\dots,X(f^{m}))=Df (X).
\]


On the other hand, if $\alpha\in\Gamma(E)$ is a section of a vector bundle
$E$, there is no natural way to differentiate $\alpha$ in the direction of a
vector field. A connection on a vector bundle $E$ is a rule that prescribes
how to differentiate sections of $E$ in the direction of vector fields.

\begin{definition}
	Let $\pi:E\rightarrow M$ be a vector bundle. A connection $\nabla$ on $E$ is a
	linear map:
	\[
	\nabla:\mathfrak{X}(M)\otimes\Gamma(E)\rightarrow\Gamma(E);\quad
	(X,\alpha)\mapsto\nabla_{X}\alpha
	\]
	such that for any smooth function $f\in C^{\infty}(M)$, $X\in\mathfrak{X}(M)$
	and $\alpha\in\Gamma(E)$ the following two conditions are satisfied:
	\begin{enumerate}
		\item \[\nabla_{fX}\alpha=f\nabla_{X}\alpha,\]
		\item \[\nabla_{X}\left(  f\alpha\right)  =\left(  X(f)\right)  \alpha
		+f\nabla_{X}\alpha.\]
	\end{enumerate}
\end{definition}

\begin{exercise}
	Show that in the case where $E=M \times\mathbb{R}^{m}$ is the trivial bundle,
	the directional derivative described above is a connection on $E$.
\end{exercise}

\begin{definition}
	If $E$ is a vector bundle with connection, we will say that a section
	$\alpha\in\Gamma(E)$ is covariantly constant if $\nabla_{X}(\alpha)=0$ for any
	vector field $X\in\mathfrak{X}(M)$.
\end{definition}

Let us now consider the case $E=TM$ and describe how a connection is expressed
in local coordinates $\varphi=(x^{1},\dots,x^{m})$. The Christoffel symbols
$\Gamma_{ij}^{k}:M\rightarrow\mathbb{R}$ are smooth functions determined by
the condition \[\nabla_{\partial_{i}}{\partial_{j}}=\sum_{k}%
\Gamma_{ij}^{k}\partial_{k}.\]
The connection $\nabla$ is
determined by the Christoffel symbols. Given vector fields
$X=\sum_{i}a^{i}\partial_{i},$ $Y=\sum_{j}b^{j}\partial_{j}$ one computes:
\begin{align*}
\nabla_{X}Y  &  =\sum_{i}a^{i}\nabla_{\partial_{i}}\left(  \sum_{j}%
b^{j}\partial_{j}\right)  =\sum_{i,j}a^{i}\nabla_{\partial_{i}}\left(
b^{j}\partial_{j}\right) \\
&  =\sum_{i,j}a^{i}\left(  \frac{\partial b^{j}}{\partial x^{i}}\partial
_{j}+b^{j}\nabla_{\partial_{i}}\partial_{j}\right) \\
&  =\sum_{i,j}a^{i}\left(  \frac{\partial b^{j}}{\partial x^{i}}\partial
_{j}+b^{j}\sum_{k}\Gamma_{ij}^{k}\partial_{k}\right) \\
&  =\sum_{i,j}a^{i}\frac{\partial b^{j}}{\partial x^{i}}\partial_{j}%
+\sum_{i,j}a^{i}b^{j}\sum_{k}\Gamma_{ij}^{k}\partial_{k}\\
&  =\sum_{k}\left(  \sum_{i}a^i\frac{\partial b^{k}}{\partial x^{i}}\partial
_{k}+\sum_{i,j}\Gamma_{ij}^{k}b^{j}a^{i}\right)  \partial_{k}.
\end{align*}


%\begin{exercise}
%Sean $X,Y$ campos vectoriales en $U$ y $\alpha,\alpha' \in \Gamma(E)$ tales que:
%\[ X(p)=X'(p); \quad \alpha (\gamma(t))= \alpha'(\gamma(t)),\]
%donde $\gamma: I \rightarrow U$ es una curva tal que:
%\[ \gamma(0)=p; \quad \gamma'(0)=X(p).\]
%Probar que para cualquier conexi\'on $\nabla$ en $E$ se tiene que:
%\[ \nabla_X \alpha (p)= \nabla_{X'} \alpha' (p).\]
%\end{exercise}


A Riemannian metric $g$ on a manifold $M$ induces a connection, called
the \emph{Levi-Civita Connection}, on the tangent bundle $TM$.

\begin{definition}
	Let $\nabla$ be a connection on $TM$. The torsion of $\nabla$ is the function
	\begin{align*}
	T: \mathfrak{X}(M) \otimes\mathfrak{X}(M) \rightarrow\mathfrak{X}(M);
	\quad(X,Y)\mapsto\nabla_{X} Y -\nabla_{Y} X - [X,Y].
	\end{align*}
	
\end{definition}

\begin{exercise} Show that given vector fields $X,Y,Z\in\mathfrak{X}(M) $, the
	torsion satisfies:
	
	\begin{itemize}
		\item Linearity with respect to functions: \[T(fX,Y)=fT(X,Y);\quad
		T(X,fY)=fT(X,Y).\]
		
		\item Skewsymmetry: \[T\left(  X,Y\right)  +T\left(  Y,X\right)  =0.\]
	\end{itemize}
	
\end{exercise}


The previous exercise implies that one can view the torsion as a tensor:

\[
T\in\Omega^{2}(M,TM)=\Gamma(\Lambda^{2}(T^{\ast}M)\otimes TM),
\]
defined by:
\[
T(p)(v,w)=\nabla_{X}Y(p)-\nabla_{Y}X(p)-[X,Y](p),
\]
for any choice of vector fields $X,Y$ such that $X(p)=v$ and $Y(p)=w$.

\begin{definition}
	A connection on $TM$ is called symmetric if its torsion is zero.
\end{definition}

\begin{exercise}
	Show that a connection $\nabla$ is symmetric if and only if for any choice of
	coordinates, the Christoffel symbols satisfy $\Gamma_{ij}^{k}=\Gamma_{ji}%
	^{k}.$
\end{exercise}

\begin{definition}
	A connection on a riemannian manifold $\left(  M,g\right)  $ is
	compatible with the metric if:
	\[
	{X}(g(Y,Z))=g\left(  \nabla_{X}Y,\text{ }Z\right)  +g\left(  Y,\text{ }%
	\nabla_{X}Z\right)  ,
	\]
	for all $X,Y,Z\in\mathfrak{X}(M).$
\end{definition}

\begin{theorem}
	[Levi-Civita]\label{4Teo3}Let $\left(  M,g\right)  $ be a riemannian
	manifold. There exists a unique symmetric connection $\nabla$ which is
	compatible with the metric. Moreover, this connection satisfies:
	\begin{align}
	g\left(  Z,\nabla_{Y}X\right)   &  =\frac{1}{2}\left(  Xg\left(  Y,Z\right)
	+Yg\left(  Z,X\right)  -Zg\left(  X,Y\right)  \right. \nonumber\\
	&  \left.  -g\left(  \left[  X,Z\right]  ,Y\right)  -g\left(  \left[
	Y,Z\right]  ,X\right)  -g\left(  \left[  X,Y\right]  ,Z\right)  \right)  .
	\label{4ec9}%
	\end{align}
	
\end{theorem}

\begin{proof}
	Any connection compatible with the metric satisfies:
	\[
	Xg\left(  Y,Z\right)  =g\left(  \nabla_{X}Y,Z\right)  +g\left(  Y,\nabla
	_{X}Z\right)  ,
	\]%
	\[
	Yg\left(  Z,X\right)  =g\left(  \nabla_{Y}Z,X\right)  +g\left(  Z,\nabla
	_{Y}X\right)
	\]
	\[
	Zg\left(  X,Y\right)  =g\left(  \nabla_{Z}X,Y\right)  +g\left(  X,\nabla
	_{Z}Y\right)  ,
	\]
	Adding the first two equations, subtracting the third and using the symmetry one obtains:
	\begin{align*}
	&  Xg\left(  Y,Z\right)  +Yg\left(  Z,X\right)  -Zg\left(  X,Y\right) \\
	&  =g\left(  \left[  X,Z\right]  ,Y\right)  +g\left(  \left[  Y,Z\right]
	,X\right)  +g\left(  \left[  X,Y\right]  ,Z\right)  +2g\left(  Z,\nabla
	_{Y}X\right)  ,
	\end{align*}
	which implies:
	\begin{align*}
	g\left(  Z,\nabla_{Y}X\right)   &  =\frac{1}{2}\left(  Xg\left(  Y,Z\right)
	+Yg\left(  Z,X\right)  -Zg\left(  X,Y\right)  \right. \\
	&  \left.  -g\left(  \left[  X,Z\right]  ,Y\right)  -g\left(  \left[
	Y,Z\right]  ,X\right)  -g\left(  \left[  X,Y\right]  ,Z\right)  \right)  .
	\end{align*}
	Since the metric is nondegenerate, this implies uniqueness.
	In order to prove existence we define  $\nabla_{Y}%
	X$
	to be the unique vector field that satisfies Equation (\ref{4ec9}). In order to prove that $\nabla$ defined in this way is a connection, the only nontrivial statement is:
	\[\nabla_X (fY)=f \nabla_X Y +X(f) Y.\]
	For this we compute:
	\begin{align*}
	g\left(  Z,\nabla_{Y}\left(  fX\right)  \right)   &  =\frac{1}{2}\left(
	fXg\left(  Y,Z\right)  +Yg\left(  Z,fX\right)  -Zg\left(  fX,Y\right)  \right.
	\\
	&  \left.  -g\left(  \left[  fX,Z\right]  ,Y\right)  -g\left(  \left[
	Y,Z\right]  ,fX\right)  -g\left(  \left[  fX,Y\right]  ,Z\right)  \right)  .
	\end{align*}
	Using the equations
	\[
	Yg\left(  Z,fX\right)  =\left(  Yf\right)  g\left(  Z,X\right)  +fYg\left(
	Z,X\right)  ,
	\]%
	\[
	Zg\left(  fX,Y\right)  =\left(  Zf\right)  g\left(  X,Y\right)  +fZg\left(
	X,Y\right)  ,
	\]%
	\[
	g\left(  \left[  fX,Z\right]  ,Y\right)  =fg\left(  \left[  X,Z\right]
	,Y\right)  -\left(  Zf\right)  g\left(  X,Y\right)
	\]
	%
	\[
	g\left(  \left[  fX,Y\right]  ,Z\right)  =fg\left(  \left[  X,Y\right]
	,Z\right)  -\left(  Yf\right)  g\left(  X,Z\right)
	\]
	we obtain:
	\begin{align*}
	g\left(  Z,\nabla_{Y}\left(  fX\right)  \right)   &  =fg\left(  Z,\nabla
	_{Y}X\right)  +\frac{1}{2}\left(  2\left(  Yf\right)  g\left(  Z,X\right)
	\right) \\
	&  =g\left(  Z,f\nabla_{Y}X+\left(  Yf\right)  X\right)  .
	\end{align*}
	We leave it as an exercise to the reader to prove that $\nabla$ is symmetric and compatible with the metric.
\end{proof}


The connection described above is called the Levi-Civita
connection \index{Levi-Civita connection} on $(M,g)$.



\section{Geodesics and the exponential map}

Here we will explain the notions of parallel transport and geodesics. It will be convenient to
first discuss some natural operations on vector bundles and connections.

\begin{exercise}
	Let $f:N\rightarrow M$ a smooth function and $\pi:E\rightarrow M$ a vector
	bundle. Show that the set $f^{\ast}(E)=\coprod_{p\in N}E_{f(p)},$ admits a
	unique structure of a vector bundle over $N$ such that:
	
	\begin{enumerate}
		\item The map $\tilde{f}:f^{\ast}(E)\rightarrow E;\quad v\in E_{f(p)}\mapsto
		v\in E_{f(p)}$ is smooth.
		
		\item The projection $\pi:f^{\ast}(E)\rightarrow M$ is given by $v\in
		E_{f(p)}\mapsto p$ .
		
		\item The diagram:
		\[
		\xymatrix{
			f^*(E) \ar[r]^{\tilde{f}} \ar[d]_{\pi}& E\ar[d]^{\pi}\\
			M \ar[r]^{f}& N}
		\]
		commutes and is a linear isomorphism on each fiber.
		
	\end{enumerate}
\end{exercise}

\begin{exercise}
	Let $\nabla$ be a connection on $\pi:E\rightarrow M$ and $f:N\rightarrow M$ a
	smooth function. Then there exists a unique connection $f^{\ast}(\nabla)$ on
	$f^{\ast}(E)$ such that for any $\alpha\in\Gamma(E)$, $X\in\mathfrak{X}(N)$
	and $Y\in\mathfrak{X}(M)$ with $Df(p)(X(p))=Y(f(p))$ the following holds:
	\begin{equation}
	f^{\ast}(\nabla)_{X}(f^{\ast}(\alpha))(p)=\nabla_{Y}(\alpha)(f(p)).
	\label{pull0}%
	\end{equation}
	
\end{exercise}



Recall that we say that a section $\alpha\in\Gamma(E)$ of a vector bundle with
connection is covariantly constant if $\nabla_{X}(\alpha)=0,$ for any vector
field $X\in\mathfrak{X}(M)$. By imposing this conditions on vector bundles
over an interval one obtains the notion of parallel transport along a path.

\begin{proposition}
	Let $\nabla$ be a connection on a vector bundle $\pi:E\rightarrow I$, where
	$I=[a,b]$ is an interval. Given a vector $v\in E_{a}$ there exists a unique
	covariantly constant section $\alpha\in\Gamma(E)$ such that $\alpha(a)=v.$
	Moreover, the function $P_{a}^{b}:E_{a}\rightarrow E_{b}$ given by $P_{a}%
	^{b}(v)=\alpha(b)$ is a linear isomorphism. The function $P_{a}^{b}$ is called
	the parallel transport of the connection $\nabla$.
\end{proposition}

\begin{proof}
	Since all vector bundles over an interval are trivializable, we may choose a frame $\{\alpha_1,\dots, \alpha_k\}$ for $E$.
	There exists a one form $\theta \in \Omega^1(I, \mathsf{End}(E))$ such that:
	\[ \nabla_X (\alpha_i)= \theta(X, \alpha_i).\]
	Let us fix $v = \sum_i \lambda_i \alpha_i(a) \in E_a$. A section $ \alpha= \sum_i f_i \alpha_i$  is covariantly constant if it satisfies the differential equation:
	\[ \sum_i \nabla_{\partial_t} (f_i \alpha_i)=0,\]
	which is equivalent to:
	\[ \sum_i \frac{\partial f_i}{\partial t} \alpha_i+ f_i\theta( \partial_t, \alpha_i)=0.\]
	The Picard-Lindel\"of theorem guarantees the existence and uniqueness of a solution of this equation. In order to show that $P_a^b$ is linear it is enough to observe that if $ \alpha$ and $\beta$ are covariantly constant, so is $\alpha + \beta $. It remains to show that  $P_a^b$ is an isomorphism. Suppose that  $v\in E_a$  is such that $P_a^b(v)=0$.
	By symmetry we know that there exists a unique section  $\alpha \in \Gamma(E)$ such that $\alpha(b)=0$. This section is the zero section and we conclude that $v=0$.
\end{proof}


\begin{definition}
	Let $\nabla$ be a connection on $\pi:E\rightarrow M$ and $\gamma
	:[a,b]\rightarrow M$ a smooth curve. The parallel transport along $\gamma$
	with respect to $\nabla$ is the linear isomorphism:
	\[
	P_{\nabla}(\gamma):E_{\gamma(a)}\rightarrow E_{\gamma(b)};\quad P_{\nabla
	}(\gamma)(v)=P_{a}^{b}(v),
	\]
	where $P_{a}^{b}$ denotes the parallel transport associated with the vector
	bundle $\gamma^{\ast}(E)$ over the interval $I=[a,b]$ with respect to the
	connection $\gamma^{\ast}(\nabla)$.
\end{definition}

\begin{lemma}
	Let $\gamma:[a,c]\rightarrow M$ be a curve and $b\in(a,c)$. Set $\mu
	=\gamma|_{[a,b]};\quad\sigma=\gamma|_{[b,c]}.$ Then $P_{\nabla}(\gamma
	)=P_{\nabla}(\sigma)\circ P_{\nabla}(\mu).$
\end{lemma}

\begin{proof}
	It is enough to observe that if $\alpha \in \Gamma(\gamma^*(E))$ is covariantly constant then $ \alpha\vert_{[a,b]}$ and $\alpha\vert_{[b,c]}$ are also covariantly constant.
\end{proof}


\begin{exercise}
	Show that parallel transport is parametrization invariant. That is, if $\nabla$ is a
	connection on $\pi:E\rightarrow M$, $\gamma:[a,b]\rightarrow M$ is a curve and
	$\varphi:[c,d]\rightarrow\lbrack a,b]$ is an orientation preserving
	diffeomorphism then $P_{\nabla}(\gamma)=P_{\nabla}(\gamma\circ\varphi).$
\end{exercise}


\begin{definition}
	Let $\nabla$ be a connection on $TM$. A curve $\gamma:[a,b]\rightarrow M$ is called a geodesic if the
	section $\gamma' \in \Gamma( \gamma^*(TM))$ is covariantly constant with respect to the
	connection $\gamma^{\ast}(\nabla)$.
\end{definition}

In local coordinates $\varphi=(x^{1},\dots,x^{m})$ where $\gamma=(u_{1}%
,\dots,u_{m})$ and $\nabla$ has Christoffel symbols $\Gamma_{ij}^{k}$ one has
$\gamma^{\prime}(t)=\sum_{i}u_{i}^{\prime}(t)\partial_{i},$
and the geodesic equation takes the form:
\begin{align*}
\gamma^{\ast}(\nabla)_{\partial_{t}}(\gamma^{\prime}(t))  &  =\sum_{i}%
\gamma^{\ast}(\nabla)_{\partial_{t}}(u_{i}^{\prime}(t)\partial_{i})\\
&  =\sum_{i}\Big(u_{i}^{\prime\prime}(t)\partial_{i}+u_{i}^{\prime}%
(t)\gamma^{\ast}(\nabla)_{\partial_{t}}\partial_{i}\Big)\\
&  =\sum_{i}\Big(u_{i}^{\prime\prime}(t)\partial_{i}+u_{i}^{\prime}(t)\sum
_{j}u_{j}^{\prime}(t)\nabla_{\partial_{j}}\partial_{i}\Big)\\
&  =\sum_{i}\Big(u_{i}^{\prime\prime}(t)\partial_{i}+u_{i}^{\prime}%
(t)\sum_{j,k}u_{j}^{\prime}(t)\Gamma_{ij}^{k}\partial_{k}\Big).
\end{align*}
We conclude that $\gamma$ is a geodesic precisely when it satisfies the system
of differential equations:
\begin{equation}
u_{i}^{\prime\prime}(t)+\sum_{j,k}u_{j}^{\prime}(t)u_{k}^{\prime}%
(t)\Gamma_{kj}^{i}=0,\,\,  \forall i.
\end{equation}


\begin{example}
	On Euclidian space $\mathbb{R}^{m}$ the Christoffel symbols are $\Gamma
	_{ij}^{k}=0,$ and therefore the differential equation for a geodesic is just
	$u_{i}^{\prime\prime}(t)=0.$ We conclude that geodesics in euclidean space are
	straight lines.
\end{example}

\begin{theorem}
	Let $\nabla$ be a connection on $TM$. Given $v\in T_{p}M$, there
	exists an interval $\left(  -\epsilon,\epsilon\right)  $ for which there is a
	unique geodesic $\gamma:\left(  -\epsilon,\epsilon\right)  \rightarrow M$ such
	that $\gamma\left(  0\right)  =p$ and $\gamma^{\prime}(0)=v$.
\end{theorem}

\begin{proof}
	Let $\varphi =(x^1,\dots, x^m)$  be local coordinates such that $\varphi(p)=0$. We  write $\gamma(t)=(u_1(t),\dots,u_m(t))$ and want to solve the system of equations:
	\[ u''_i(t) + \sum_{j,k} u'_j(t) u'_k(t) \Gamma_{kj}^i  =0.\]
	This is a second order ordinary differential equation. The existence and uniqueness of solutions is guaranteed by the Pickard-Lindel\"of theorem.
\end{proof}


\begin{definition}
	Let $\nabla$ be a connection on $TM$ and $ p \in M$. We define $A_p \subseteq T_pM$ as follows:
	\[\ A_p:= \{ v\in T_pM: \text{there exists a geodesic } \gamma_v: [-1,1] \rightarrow M, \text{ with } \gamma_v(0)=p \text{ and }  \gamma_v'(0)=v\}.\]
	The exponential map is defined by:
	\[ \mathsf{Exp_p}: A_p \rightarrow M; \, v \mapsto \gamma_v(1).\]
\end{definition}
The proof of the following theorem can be found in any text on riemannian geometry, for example \cite{riemannian}.
\begin{theorem}
	Let $\nabla$ be a connection on $TM$ and $ p \in M$. The domain $A_p$ of the exponential map contains an open
	neighborhood around $0 \in T_pM$. Moreover, the derivative of the exponential map at $0$ is the identity and therefore the exponential map is a local diffeomorphism.
\end{theorem}


\section{Principal bundle}
	\begin{definition}
	An action of a group  $G$ on a manifold  $M$ is a smooth function 
	\begin{align*}
	\mu &: M \times G  \longrightarrow M \\
	&\quad (m,g) \longmapsto m\cdot g
	\end{align*}
	such that:
	$m \cdot e=m$ y $(m \cdot g)h=m\cdot(gh)$. In particular, $\mu$ is a called free action if, for all $m\in M$, $mg=m$ implies $g=e$.
\end{definition}


A principal $G$-bundle, where $G$ is any Lie group, is a surjective submersion $\pi: P \to M$ equipped with smooth action of $G$ on $P$ that satisfies:
\begin{enumerate}[(i)]
	\item The action is free.
	\item $\pi(mg)=\pi(m)$.
	\item If $\pi(m)=\pi(m')$ then $m=m'g$.
\end{enumerate}
Since $\pi$ is a submersion then for all $q\in M$, $ \pi^{-1}(q)$ is a manifold of dimension \linebreak{$d=dim(P)-dim(M)$}.\\

\subsection{Some properties of principal bundle}
\begin{theorem}
	Every fiber of the principal $G$-bundle is diffeomorphic to $G$.
\end{theorem}
\begin{proof} The following function is a diffeomorphism
	\begin{align*}
	\psi &: G \longrightarrow \pi^{-1}(q) \\
	&\quad g \longmapsto pg
	\end{align*}
\end{proof}


		\begin{theorem}
	Let $\pi: P \to M $ be a principal  $G$-bundle, then there are a natural identification $P/G  \xrightarrow[\tilde{\pi}]{\sim} M$, in particular $P/G \cong M$.
\end{theorem}

A priori $P/G$ has structure of the topological space, then we give to $P/G$ differential structure induced by $\tilde{\pi}$.

\begin{theorem} Every principal bundle is locally isomorphic to the trivial bundle.
\end{theorem}
\begin{proof}
Let  $P \to M$ be a principal bundle, then for every $q \in M$ there exists a open \linebreak{$q \in U \subseteq M$} and  $\sigma: U \to P$ such that $\pi \circ \sigma=id$.\\
The following function is a diffeomorphism:
	$$\begin{array}{crcl}
\tau &: U \times G & \longrightarrow & \pi^{-1}(U) \\
& (x,g) & \mapsto & \sigma(x)\cdot g
\end{array}$$
\end{proof}

\subsection{Frame bundles and Associated bundle}
If $\pi:E \to M$ is a complex vector bundle over $M$ of rank $k$, then the geometry structure that contains information about the set of all ordered bases (or frames) for a fiber of $E$ is called frame bundle, more specifically, a frame bundle is a principal bundle with structural group $GL(\mathbb{C}^{k})$:

	$$\begin{array}{crcl}
\rho &: Fr(E) & \longrightarrow & M \\
& \varphi & \mapsto & p
\end{array}$$

Where $\displaystyle{Fr(E):=\coprod_{p\in M} ISO(\mathbb{C}^{k},E_{p})}$, $\varphi: \mathbb{C}^{k} \to E_{p}$ is a isomorphism and the action of $GL(\mathbb{C}^{k})$ on $Fr(E)$ is given by conjugation.\\

If a vector bundle $E$ is equipped with a Riemannian bundle metric, then the orthonormal frame bundle of $E$ is $\displaystyle{Fr{O}(E):=\coprod_{p\in M} ISO^{O}(\mathbb{R}^{k},E_{p})}$, the set of all orthonormal frames at each fiber with structural group $O(k)$.\\
In similar form, if a vector bundle $E$ is equipped with a hermitian bundle metric, then the hermitian frame bundle is  $\displaystyle{Fr{U}(E):=\coprod_{p\in M} ISO^{U}(\mathbb{C}^{k},E_{p})}$ with structural group $U(k)$.\\

Let us now consider a principal $G$-bundle $\pi :P \to M$ and a representation $\rho :G \to Aut(V)$ of group $G$, then it is possible to construct a vector bundle over $M$, denoted associated bundle.\\
Note that $G$ acts on $P \times V$, for $g \in G$, $(p,v) \in P\times V$, $(p,v) \cdot g= (p\cdot g, \rho(g^{-1})(v))$, then $P \times_{G} V:= P \times V/G$ where $(p\cdot g,v) \sim (p,\rho(g)(v))$.\\
The following submersion $\tilde{\pi}:P \times_{G} V \to M$  is a vector bundle:
	$$\begin{array}{crcl}
\tilde{\pi} &: P \times_{G} V & \longrightarrow & M \\
& (p,v) & \mapsto & \pi(p)
\end{array}$$

When the structural group is  $GL(n,\mathbb{C})$ and the representation is $id$, the construction of associated bundle is inverse to it of frame bundle. 

\section{Connection on a principal bundle}
\begin{definition}
	A distribution $E$ on $M$ is a subbundle of the tangent bundle and if  \linebreak{$\Upgamma(E) \subseteq \Upgamma(TM)$} is a Lie subalgebra then $E$ is called foliation. 
\end{definition}
	\begin{definition}
	Let  $\pi: X \to Y$ be a submersion,then  a Ehresmann connection is a distribution $\mathcal{H} \subseteq TX$ such that every $p\in X$, $\mathcal{H}(p) \oplus Ker(d\pi)=TX$.
\end{definition}
\begin{theorem}
	$Ker(d\pi)$ is a foliation.
\end{theorem}


\begin{definition}
	Let $P \to M$ be a principal $G$-bundle over a manifold $M$, a connection on $P$ is a Eheresmann connection  which is $G$-equivariant, in other words, for $q=pg$, $D(R_{g})(p)(H_{p})=H_{q}$.
\end{definition}

\begin{theorem}
	There exists an isomosphism between the bundles $Ker(d\pi)$ and $P \times \mathfrak{g}$.
\end{theorem}
\begin{proof}
	Since that  $P \to M$ is a principal  $G$-bundle, there exists a function: 
	$$\begin{array}{crcl}
	\mu: & \mathfrak{g} & \longrightarrow & \mathfrak{X}(P) \\
	& v & \mapsto & v^{*}
	\end{array}$$
	Where  $v^{*}(p)=\frac{d}{dt} \big |_{t=0} p\cdot exp(tv)$.\\
	It is easy to check that  $\mu$ is a Lie algebras homomorphism. \\
	In fact the following function is linear isomorphism:
	 $$\begin{array}{crcl}
	\mu_{p} :& \mathfrak{g} & \longrightarrow & Ker(d\pi(p)) \\
	& v & \mapsto & v^{*}(p)
	\end{array}$$
	Note that $dim(\mathfrak{g})=dim(Ker(d\pi))$, then prove that $\mu_{p}$ is a injective function is enough to prove that it is bijective.
	Let $\mu_{p}(v)=0=v^{*}(p)=\frac{d}{dt}\big|_{t=0} p\cdot exp(tv) $. \\
	\begin{align*}
	\frac{d}{dt}\big|_{t=t_{0}} p\cdot exp(tv)&=\frac{d}{dz}\big|_{z=0} p\cdot exp((z+t_{0})v)\\&=\frac{d}{dz}\big|_{z=0} p\cdot exp(tv) \cdot exp(t_{0}v)\\&=D\left(R_{exp(t_{0}v)}\right) \left(\frac{d}{dz}\big|_{z=0} p\cdot exp(zv)\right)\\&=D(R_{exp(t_{0}v)})(0)=0
	\end{align*}\\
	Then $v=0$.
\end{proof}

To continue, we identify a connection $\mathcal{H}$ on a principal bundle $P$ with a $1$-form Lie algebra-valued form on $P$, with the aim of making calculations  of connections more easy.

\begin{theorem} 
	Let $\mathcal{H} \subseteq TP$ be a connection on $P$ then for each $p \in P$, $T_{p}P= \mathfrak{g} \oplus T_{\pi(p)}M$. 
\end{theorem}
\begin{proof}
	Previous results have demonstrated that:
	\begin{align*}
	T_{p}P=Ker(d\pi(p))\oplus \mathcal{H}_{p} \quad y \quad Ker(d\pi(p))=\mathfrak{g}
	\end{align*} 
	Furthermore the following function is a linear isomorphism
	\begin{align*}
	d\pi_{p}: \mathcal{H}_{p} \xrightarrow{\sim} T_{\pi(p)}M
	\end{align*}
\end{proof}
 Therefore if $\mathcal{H}\subseteq TP$  is a connection on $P$ then there exits a $1$-form $\theta \in \Upomega^{1}(P,\mathfrak{g})$ that satisfy:
 \begin{enumerate}[$\cdot$]
 	\item $\theta(v^{*}(p))=v$.
 	\item $\theta$ is equivariant.
 \end{enumerate}
 The $1$-form $\theta$ is defined as $\theta(p)(x)=v$, where $v \in \mathfrak{g}$ is the unique vector such that $v*(p)=x^{v}$ in the decomposition $T_{p}P= \mathfrak{g} \oplus T_{\pi(p)}M$, for $x \in T_{p}P$, $x=x^{v}+x^{h}$.

Conversely if $\theta \in \Omega^{1}(P,\mathfrak{g})$ is a $1$-form equivariant such that $\theta(v^{*}(p))=p$ then there exists a connection in $P$ given by $Ker(\theta)$.


\section{Curvatute on a principal bundle}
\begin{definition}
	A horizontal form of a principal bundle $\pi: P \to M$ is a form $\theta \in \Omega(M)$ such that $\theta(p)(x_1,x_2,...,x_n)=0$ if any $x_i \in Ker(d\pi)$
\end{definition}
\begin{definition}
	A differential graded  Lie algebra (dgla) is a graded vector space $\mathfrak{g} =\oplus_{k \geqslant 0} \mathfrak{g}^{k}$ with added Lie algebra and chain complex structures that are compatible, i.e. is a graded vector space together with a linear map $[,]:\mathfrak{g}^{i} \otimes \mathfrak{g}^{j} \to \mathfrak{g}^{i+j} $ and a differential map $\delta: \mathfrak{g}^{k}\to \mathfrak{g}^{k+1}$ satisfying:
	\begin{enumerate}[i]
		\item $[v,w]+(-1)^{|v||w|}[w,v] =0$ (graded skew-symmetric)
		\item $[v,[w,z]]=[[v,w],z]+(-1)^{|v||w|}[w,[v,z]]$ (Jacobi Identity)
		\item $\delta([v,w])=[\delta v, w]+(-1)^{|v|}[v, \delta w]$ (graded Leibniz rule)
		\item $\delta \circ \delta=0$
	\end{enumerate}
\end{definition}

Since $\Omega^{even}(P)$ is a commutative differential graded algebra and $\mathfrak{g}$ a Lie algebra, \linebreak{ $\Omega(P, \mathfrak{g}) \cong \Omega(P)\otimes \mathfrak{g} $} is a dgla together with the maps: $\delta(\theta \otimes v)=d\theta \otimes v$ and \linebreak{$[\theta \otimes v, \omega \otimes w]= \theta \omega  \otimes [v,w]$}.\\

 Let $\mathfrak{h}$ be a dgla and a connection $1$-form $\theta \in \Omega^{1}(P, \mathfrak{g})$, then the curvature of $\theta$, $F_{\theta}$ is a Lie algbra-valued $2$-form defined by: $F_{\theta}=d\theta+\frac{1}{2}[\theta,\theta]$.\\
 A manifold is said to be flat if its curvature is everywhere zero.\\
 
 The curvature on a principal bundle  satisfies the following properties:
 \begin{enumerate}[i]
 	\item $d(F_{\theta})+[\theta,F_{\theta}]=0$ (Bianchi Identity)
 	\item $F_{\theta}$ is a equivariant and horizontal form.
 	\item $\theta$ is a foliation if only if M is a flat manifold.
 
  \end{enumerate}

\section{Chern-Weil Homomorphism}
The purpose of this section in to give a brief introduction to the Chern-Weil theory of characteristic classes, which was developed by Shiing-Shen Chern and André Weil  in the first half of the 20th century. For each principal $G$-bundle over a manifold $M$ we want to construct a homomorphism from $G$- invariant polynomials to cohomology of $M$.\\
Let $\pi: P \to M$ be a principal $G$-bundle over M together with a connection $\theta$, then the curvature $F_{\theta}$ determines a homomorphism of algebras given by:
 $$\begin{array}{crcl}
\Phi :& S(\mathfrak{g}^{*}) \quad & \longrightarrow & \Omega(P) \\
& \alpha_{1} \odot \dots \odot \alpha_{k} & \mapsto & \phi(\alpha_{1}) \wedge \dots \wedge \phi(\alpha_{k}).
\end{array}$$
Where $\Phi(\alpha_{i})(X,Y)=\alpha_{i}(F_{\theta}(X,Y))$ and $S(\mathfrak{g}^{*}) $ denote the symmetric algebra of $\mathfrak{g}^{*}$.

\begin{theorem}
	Let $\pi: P \to M$ be a principal $G$-bundle together with a connection $\theta$, then the homomorphism $\phi$ defined above satisfies:
	\begin{enumerate}
	\item $\Phi(S(\mathfrak{g}^{*})^{G}) \subseteq \Omega(P)^{G}$.
	\item Each form of  $\Phi(S(\mathfrak{g})^{*})$ is horizontal.
	\item $\Omega^{bas}(P) \cong \Omega(M)$.
	\item $d(\Phi(S(\mathfrak{g}^{*})))=0$. 
	\end{enumerate}
Here $\Omega^{bas}(P)$ denote the set of horizontal and equivariant forms of $P$, it is called basic forms.
\end{theorem}


\begin{proof}
\begin{enumerate}
	\item We will prove that $\Phi: S(\mathfrak{g}^{*})^{G} \to \Omega(P)$ is equivariant, in particular \linebreak{$\Phi(S(\mathfrak{g}^{*})^{G}) \subseteq \Omega(P)^{G}$}.\\
	$\Phi(g \cdot \alpha)(X,Y)=(g \cdot \alpha )(F(X,Y))= \alpha \circ Ad(g^{-1})F(X,Y)=\alpha(F(DR_{g}X,DR_{g}Y))=g \cdot(\Phi(\alpha))(X,Y)$
	\item It is enough to note that for $\varphi \in \mathfrak{g}^{*}$, $\Phi(\varphi)$ is horizontal. Then for any $X$ vertical, $\Phi(\varphi)(X,Y)=\varphi(F(X,Y))=\varphi(0)=0$.
	\item Consider the principal $G$-bundle $\pi: P \to M$ then there exist a algebra homomorphism $\pi^{*}: \Omega(M) \to \Omega(P)$. In fact $\pi^{*}$ is injective because if $\pi^{*}(\theta)(p)(X_{1},...,X_{k})=0$ for every $p\in P$ and $X_{1},...,X_{k} \in T_{p}P$, $\theta(\pi(p))(D\pi(X_{1},...,D\pi(X_{k})))=0$ then  $\theta(\pi(p))(Y_{1},...,Y_{k})=0$ every $\pi(p) \in M$ and $Y_{1},...,Y_{k} \in T_{\pi} \in T_{\pi(p)}M$.\\
	The homomorphism $\pi^{*}$ is surjective when we restrict the codomain to the set of horizontal and equivariant forms. The surjective of this is verified with the following lemma when we consider the trivial representation.\\
	\begin{lemma}
		Let $\pi: P \to M$ be a principal $G$-bundle and a representation of $G$ $\rho:G \to GL(V)$ then there the horizontal and equivariant forms of $\Omega(P,V)$ are in a one to one correspondence with forms of $\Omega(M,\rho(P))$, where $\rho(P)$ denotes the bundle $P\times_{G}V$.
	\end{lemma}
	\begin{proof}
		Consider the function:
$$\begin{array}{crclccc}
& \left\{\begin{array}{cc}
& \omega \in \Omega(P,V) \quad \text{such that} \\
& \omega \quad \text{is horizontal}\\
& \text{and equivariant}
\end{array}
\right\}    & \xrightarrow{\psi} & \Omega(M,\rho(P)); &\theta & \mapsto & \psi(\theta).
\end{array}$$
Defined by $\psi(\theta)(y)(X_{1},...,X_{k})=[(p, \theta(p)(\tilde{X_{1}},...,\tilde{X_{k}})]$ where $p \in \pi^{-1}(y)$ and \linebreak{ $D\pi(\tilde{X_{i}})=X_{i}$}. The well defined of this function is given by the condition of $\theta$ be a horizontal and equivariant form.  Note that the pullback of differential forms is the inverse function to this $\psi$.
 

	\end{proof}

\item Let $\varphi \in \mathfrak{g}^{*}$ and $X_{1},X_{2},X_{3}$ horizontal vectors, then 
$$d\big(\Phi(\varphi)\big)(X_1,X_2,X_3)=\varphi\big(dF(X_1,X_2,X_3)\big)=-\dfrac{1}{2}\varphi\big([\theta,F](X_1,X_2,X_3)\big)$$
$$=-\dfrac{1}{2}\varphi\big([\cancelto{\mbox{\footnotesize 0}}{\theta(X_1)},~\, F(X_2,X_3)]-[\can{\theta(X_3)},~\, F(X_1,X_2)]+[\can{\theta(X_2)},~\, F(X_3,X_1)]\big)=0$$
By induction, it follows  $d\big(\Phi(\varphi_1)\wedge\dots\wedge \Phi(\varphi_k)\big)=0$.

\end{enumerate}
\end{proof}

In the following steps we will construct the Chern- Weil homomorphism:
\begin{itemize}
\item Given a connection $\theta$ of a principal $G$-bundle, its curvature $F_{\theta}$ there exists a algebras homomorphism  $S(\mathfrak{g}^*)\overset{\Phi}{\longrightarrow}\Omega(P)$.
\item We proved that this homomorphism is equivariant and its image is a set of basic forms.
\item Use the isomorphism $\Omega(P)^{bas}\cong\Omega(M)$ to extender $\Phi$:
\begin{center}
	\begin{tikzcd}[cramped, row sep=0.1cm, column sep=0.5 cm]
		& S(\mathfrak{g}^*)\ar[r]\ar[rr,"\Phi", bend left]&\Omega(P)^{bas}\ar[r, "\cong" '] & \Omega(M).
	\end{tikzcd}
\end{center}
\item By the last theorem, the image of $\Phi$ is a set of the closed forms in $\Omega(M)$ and therefore we can extend the function $\Phi$ to the De Rham cohomology of $M$ via natural projection: 
\begin{center}
	\begin{tikzcd}[cramped, row sep=0.1cm, column sep=0.5 cm]
		&S(\mathfrak{g}^*)\ar[r]\ar[rrr,"\Phi", bend left]&\Omega(P)^{bas}\ar[r, "\cong" ']&\Omega(M)\ar[r]&H^*(M).
	\end{tikzcd}
\end{center}
\item Finally we restrict the domain of $\Phi$ to $S(\mathfrak{g}^*)^G$ and we obtain the Chern-Weil homomorphism:
$$\Phi: S(\mathfrak{g}^{*})^{G} \to H^{*}(M)$$.
\end{itemize}


\section{Curvature and the Chern-Gauss-Bonnet Theorem}

The Gauss-Bonnet theorem provides a formula for the Euler characteristic of a closed oriented surface $\Sigma$:
\[ \chi(\Sigma)= \frac{1}{2\pi} \int_\sigma K dA.\]

Here $K$ denotes the Gaussian curvature of $\Sigma$ associated to a riemannian metric and $dA$ is the volume form determined by the metric and the orientation. This formula is remarkable because while the left hand side evidently depends only on the topology of $\Sigma$, the right hand side is a priori a geometric quantity. The extension of this formula to higher dimensions had to wait almost 200 years until the language of differential geometry was developed and Chern \cite{Chern} proved his generalised version of the Gauss-Bonnet theorem. The first obstruction that needs to be overcome in order to state a correct generelisation is to find a replacement for the gaussian curvature. This is provided by the Riemann curvature tensor.

\begin{definition}
	Let $\nabla$ be a connection on $TM$. The riemannian curvature of $\nabla$ is the map:
	\[ R: \mathfrak{X}(M) \otimes \mathfrak{X}(M) \otimes \mathfrak{X}(M) \rightarrow \mathfrak{X}(M),\]
	given by:
	\[ R_\nabla (X,Y,Z):= \nabla_X (\nabla_Y Z)- \nabla_Y (\nabla_X Z)- \nabla_{[X,Y]}Z.\]
	A simple computation shows that this map is $C^\infty(M)$-linear in all the components and skew symmetric in $X$ and $Y$. Therefore, it defines a tensor:
	\[ R_\nabla \in \Omega^2(M, \mathsf{End}(TM))\]
	which is called the Riemann curvature tensor. A connection $\nabla$ is said to be flat if $R_\nabla=0.$
\end{definition}
 Mostrar la relación que hay entre la dos forma, curvatura definida para un G fibrado principal en general yla definición anterior, las definiciones son compatibles. 
 
 En el que caso que una superficie admita la metrica de LEvi-Civita, se cumple la conjetura por la siguiente observación:
\begin{exercise}
	Show that if $\nabla$ is the Levi-Civita connection of a riemannian manifold then for any pair of tangent vectors
	$v,w \in T_pM$, the map:
	\[ R_\nabla(v,w,-): T_pM \rightarrow T_pM\]
	is antisymmetric, i.e.:
	\[  \langle R_\nabla(v,w,z), z \rangle =0.\]
	This is one of the Bianchi identities.
\end{exercise}

Enunciar el teorema de chern gauss bonnet en su forma general, lo que hay abajo.

\begin{exercise} Let $ \mathfrak{so}(2n) $ be the Lie algebra of skew symmetric matrices.
	The Pfaffian polynomial: \[\mathsf{Pf}: \mathfrak{so}(2n) \rightarrow \R,\] is defined by the formula:
	
	\[ \mathsf{Pf}(A)= \frac{1}{n!2^n }\sum_{\sigma \in S_{2n}} sg(\sigma) \prod_{i=1}^n a_{\sigma(2i-1),\sigma(2i)}.\]
	Show that for $B \in \mathsf{End}(\R^{2n})$ and $ A \in \mathfrak{so}(2n)$ the following holds:
	\begin{itemize}
		\item[(a)] \[ \mathsf{Pf}(B A B^t)= \det(B) \mathsf{Pf}(A).\]
		\item[(b)] \[ \mathsf{Pf}(A)^2= \det(A).\]
	\end{itemize}
	In particular, if $B \in SO(n)$ then:
	\[ \mathsf{Pf}(B A B^{-1})= \mathsf{Pf}(A).\]
\end{exercise}


\begin{exercise}
	Let $V$ be a real vector space of dimension $2n$ and $ \langle \,, \rangle$ an inner product in $V$. Let $\mathfrak{so}(V)$ be the space of antisymmetric endomorphisms of $V$. That is:
	\[ \mathfrak{so}(V):=\{ \phi: V \rightarrow V: \langle v , \phi(v)\rangle=0\}.\]
	Let $\mathsf{P}(A)=\mathsf{P}(a_{ij}): \mathfrak{so}(2n) \rightarrow \R$ be a polynomial which is invariant under the action of $SO(2n)$ i.e. such that:
	\[ P(BAB^{-1})=P(A)\]
	for all $B \in SO(2n)$. Fix an orthonormal basis $\{ e_1, \dots, e_{2n}\}$ for $V$. For any $\omega \in \Lambda^2 (V^*) \otimes \mathfrak{so}(V)$ define $\omega_{ij} \in \Lambda^2(V^*)$ by the formula:
	\[ \omega_{ij}(v,w):=\langle \omega(v,w)(e_i), e_j \rangle.\]
	Show that $\mathsf{P}(\omega):= \mathsf{P}(\omega_{ij})\in \Lambda(V^*)$ does not depend on the choice of orthonormal basis.
\end{exercise}


The previous exercise shows that for any riemannian manifold $(M,g)$ of dimension $2n$ there is a well defined form $\mathsf{Pf}(K)\in \Omega^{2n}(M)$ which is defined by:
\[ \mathsf{Pf}(K)(p):= \mathsf{Pf}(K(p))\in \Lambda^{2n}(T^*_pM).\]
This differential form is what needs to be integrated over $M$ to obtain the Euler characteristic:

\begin{theorem}[Chern-Gauss-Bonnet]
	Let $(M,g)$ be a closed oriented riemannian manifold of dimension $d=2m$ and $R$ the curvature of the Levi-Civita connection. Then:
	\[ \chi(M)=\Big(\frac{1}{2\pi}\Big)^n \int_M \mathsf{Pf}(K).\]
\end{theorem}

Of course, in case the dimension of $M$ is odd, the Euler characteristic vanishes by Poincar\'e duality.