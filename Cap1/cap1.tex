An affine structure on a manifold is an atlas whose transition functions are affine transformations. The existence of such a structure is equivalent to the existence of a flat torsion free connection on the tangent bundle. Chern's conjecture states the following:

\begin{conjecture}[Chern $\sim$ 1955]
	The Euler characteristic of a closed affine manifold is zero.
\end{conjecture}
In case the connection $\nabla$ is the Levi-Civita connection of a riemannian metric, the Chern-Gauss-Bonnet formula:

\[ \chi(M)= \Big(\frac{1}{2\pi}\Big)^n \int_M \mathsf{Pf}(K),\]

\noindent implies that the Euler characteristic is zero. However, not all flat torsion free connections on $TM$ admit a compatible metric, and therefore, Chern-Weil theory cannot be used in general to write down the Euler class in terms of the curvature.\\

These thesis contain an exposition of the following results concerning  the Euler characteristic of affine manifolds:
\begin{itemize}
	\item In 1955, Benz\'ecri \cite{B} proved that a closed affine surface has zero Euler characteristic.
	
	\item In 1958, Milnor \cite{Milnor} proved inequalities which completely characterise those oriented rank two bundles over a surface that admit a flat connection.
	
	\item In 1975, Kostant and Sullivan \cite{KS} proved Chern's conjecture in the case where the manifold is complete.
	
	\item In 1977, Smillie \cite{Smillie} proved that the condition that the connection is torsion free matters. For each even dimension greater than $2$, Smillie constructed closed manifolds with non-zero Euler characteristic that admit a flat connection on their tangent bundle.
	
	
	
	\item In 2015, Klingler \cite{K} proved the conjecture for special affine manifolds. That is, affine manifolds that admit a parallel volume form.
	
\end{itemize}

Recently, Feng and Zhang \cite{FZ} posted a proof of the general case using the Mathai-Quillen formalism for characteristic classes. Due to lack of understanding, we will not cover their work.\\ \\
We have added appendices with preliminary material.
Appendix A contains a review of the theory of connections, curvature and the Chern-Gauss-Bonnet theorem.
Appendix B contains introductions to spectral sequences and Appendix C to sheaf theory.
\section{Affine manifolds}


\begin{definition}
	A diffeomorphism $\varphi$ between open subsets of $\R^m$ is affine if it has the form:
	\[ \varphi(x)= Ax+b,\]
	where $A \in {GL}(m,\R)$ and $b \in \R^m$.
\end{definition}

\begin{definition}
	An \index{Afffine structure} affine structure on a manifold is an atlas such that all transition functions are affine and it is maximal with this property.
	An affine manifold is a manifold together with an affine structure.
\end{definition}

It is possible to characterise affine structures in a more intrinsic manner that can be expressed without reference to an atlas, as the following lemma shows:

\begin{lemma}
	Let $M$ be a manifold. There is a natural bijective correspondence between affine structures on $M$ and flat torsion free connections on $TM$.
\end{lemma}
\begin{proof}
	Let $(U_\alpha,\varphi_\alpha)$ be an affine structure on $M$. There is a unique connection $\nabla$ on $TM$ whose Christoffel symbols vanish in affine coordinates. Conversely, given a flat torsion free connection $\nabla$ on $TM$, the set of coordinates for which the Christoffel symbols of $\nabla$ vanish gives an affine structure on $M$.
\end{proof}

\begin{example}
	The torus $\mathbb{T}^m:= \R^m/\mathbb{Z}^m$ has a natural affine structure for which the projection map $\pi: \R^m \rightarrow \mathbb{T}^m$ is an affine local difeomorphism.
\end{example}


\begin{example}[Hopf manifolds]
	Let us fix a real number $\lambda>1$ and consider the action of the group $\mathbb{Z}$ on $\R^m-\{0\}$ given by:
	\[ n \star x:= \lambda^n x.\]
	Since the action is free and proper, the quotient is a smooth manifold called the Hopf manifold $\mathsf{Hopf}^m_\lambda$. Since the group $\mathbb{Z}$ acts by affine transformations the quotient space is an affine manifold. Topologically, these manifolds are the union of two circles for $m=1$ and diffeomorphic to $S^{m-1}\times S^1$ for $m>1$.
\end{example}


\begin{definition}
	An affine structure on a Lie group $G$ is called left invariant if for all $g \in G$:
	\[ (L_g)^*(\nabla)=\nabla,\]
	where $L_g$ denotes the diffeomorphism given by left multiplication by $g$.
\end{definition}

\begin{definition}
	Let $V$ be a real vector space. A bilinear map:
	\[\beta: V \otimes V \rightarrow V;\,\, v \otimes w \mapsto v\cdot w\]
	is called left symmetric if:
	\[ v\cdot (w \cdot z)-(v\cdot w) \cdot z= w\cdot (v \cdot z)-(w\cdot v)\cdot z.\]
\end{definition}


\begin{definition}
	An affine structure on a finite dimensional real Lie algebra $\mathfrak{g}$ is a left symmetric bilinear form on $\mathfrak{g}$ such that for all $v,w \in \mathfrak{g}$:
	\[ [v,w]=v\cdot w- w \cdot v.\]
\end{definition}

\begin{lemma}
	Let $G$ be a Lie group with Lie algebra $\mathfrak{g}$. There is a natural bijective correspondence between
	left invariant affine structures on $G$ and affine structures on $\mathfrak{g}$.
\end{lemma}

\begin{proof}
	For any $v \in \mathfrak{g}=T_eG$ denote by $\hat{v}$ the corresponding left invariant vector field. Given a left invariant affine structure on $G$ we define a bilinear form on $\mathfrak{g}$ by:
	\[ v\cdot w:= \nabla_{\hat v}\hat{w}(e).\]
	Since $\nabla$ is torsion free we know that:
	\[ v \cdot w - w \cdot v =  \nabla_{\hat v}\hat{w}(e)-  \nabla_{\hat w}\hat{v}(e) =[v,w].\]
	Using the fact that $\nabla$ is flat and left invariant, we compute:
	\begin{eqnarray*}
		v\cdot (w \cdot z)-(v\cdot w) \cdot z-w\cdot (v \cdot z)+(w\cdot v)\cdot z&=& \nabla_{\hat{v}} (\nabla_{\hat{w}}(\hat{z}))\\
		&\quad&-\nabla_{\nabla_{\hat{v}}(\hat{w})}(\hat{z}) - \nabla_{\hat{w}} (\nabla_{\hat{v}}(\hat{z}))+\nabla_{\nabla_{\hat{w}}(\hat{v})}(\hat{z})\\
		&=&\nabla_{[\hat{v},\hat{w}]}(\hat{z})-\nabla_{[\hat{v},\hat{w}]}(\hat{z})\\
		&=&0.
	\end{eqnarray*}
	So we conclude that the bilinear form is left symmetric. Conversely, given an affine structure on $\frak{g}$ there is a unique left invariant connection $\nabla$ such that:
	\[ \nabla_{\hat{v}}\hat{w}(e)=v\cdot w.\]
	The computations above show that this connection is flat and torsion free.
\end{proof}

\begin{example}
	The Lie algebra $\mathfrak{gl}(n,\R)$ admits a natural affine structure given by:
	\[ A \cdot B:= AB.\]
	We conclude that the Lie group $GL(n, \R)$ admits a left invariant affine structure.
\end{example}


The question of which Lie groups admit left invariant affine structures is an interesting problem. In \cite{Milnor2}, Milnor asked whether solvable Lie algebras admit affine structures. In \cite{Bu}, Burde showed that the answer to this question is negative.

\section{Complete manifolds and the developing map}

In this section we introduce the developing map of a simply connected affine manifold and use it to characterise complete affine manifolds.

\begin{theorem}
	Let $M$ be an affine manfiold of dimension $m$ and $G$ be the group $\mathsf{Aff}(\mathbb{R}^m)$ seen as a discrete group. There is a natural principal $G$-bundle \linebreak{ $\pi:\tau(M) \rightarrow M$} such that sections  of $\pi$ are in natural bijective correspondence with affine immersions from $M$ to $\R^m$. \end{theorem}
\begin{proof} For each $p\in M$ we define:
	\[C_{p}= \left\{  \varphi: U\to V\subseteq \R^m:  \varphi \text{ is an affine diffeomorphism and }p\in U \right\}.\]
	
	There is an equivalence relation $\sim$ on $C_p$ given by declaring that $\varphi \sim \varphi'$ if an only if there exists an open subset $W \subseteq U \cap U'$ such that $p \in W$ and $\varphi\vert_W=\varphi' \lvert_W$. Let us denote by $L_p$ the set of equivalence classes of elements in $C_p$ and set:
	\[\tau(M):=\coprod_{p} L_{p}.\]
	There is a natural map:
	\[
	\pi:\tau(M) \rightarrow M . \]
	The Lie group $G$ acts on each set $L_p$ by composition:
	\[g \ast \varphi := g \circ \varphi,\]
	and this action is free and transitive. Moreover, an affine chart $\varphi: U \rightarrow V\subseteq \R^m$ induces a natural identification:
	\[ \hat{\varphi}: \mathsf{Aff}(\R^m)\times U \rightarrow \pi^{-1}(U); \,\, (g,p)\mapsto [ g \circ \varphi]_p, \]
	where $[g \circ \varphi]_p$ denotes the class of $g \circ \varphi$ in $L_p$. There is a unique topology on $\tau(M)$ such that $\hat{\varphi}$ is a homeomorphism for all affine diffeomorphisms $\varphi$. The map $\pi: \tau(M) \rightarrow M$ is a principal $G$ bundle with respect to this topology. Let $\sigma$ be a section of $\pi$. Then we define $\tilde{\sigma}:M \rightarrow \R^m$ by:
	\[ \tilde{\sigma}(p):= \sigma(p) (p).\]
	By construction, the map $\tilde{\sigma}$ is an affine immersion. Conversely, an affine immersion \linebreak{ $f: M \rightarrow \R^m$} defines a section $\sigma_f$ given by:
	\[ \sigma_f(p):= [f]_p.\]
\end{proof}


\begin{corollary}\label{extend}
	Let $M$ be a simply connected affine manifold. Any affine chart \linebreak{ $\varphi: U \rightarrow V \subseteq \R^m$} extends uniquely to an affine immersion from $M$ to $\R^m$. \end{corollary}

\begin{proof}
	Since $M$ is simply connected, the covering space $\pi: \tau(M) \rightarrow M$ is trivial and therefore any local section extends uniquely to a global one.
\end{proof}

\begin{definition} \index{Developing map}
	A developing map for an affine simply connected manifold is an affine immersion into $\R^m$.
\end{definition}

\begin{corollary}
	If $M$ is an affine connected manifold with finite fundamental group then $M$ is not compact.
\end{corollary}

\begin{proof}
	If $M$ is compact then so is its universal cover $\tilde{M}$ which is simply connected and therefore admits an immersion to $\R^m$. This is impossible.
\end{proof}

\begin{definition} \index{Complete manifold}
	An affine manifold $M$ with affine connection $\nabla$ is called complete if all geodesics can be extended to arbitrary time.
\end{definition}

\begin{lemma}\label{unique}
	Let $M$ and $N$ be connected affine manifolds of the same dimension and $f,g:M \rightarrow N$ affine immersions. If $f$ and $g$ coincide on a nonempty open subset of $M$ then they are equal.
\end{lemma}
\begin{proof}
	First we observe that the lemma is true if $M$ and $N$ are open subsets of $\R^m$.
	For the general case, let $X\subset M$ be the subset of $M$ that consists of points $p\in M$ such that $f$ and $g$ coincide in an open neighborhood of $p$. Clearly, $X$ is open. It suffices to show that $X$ is closed.
	Fix $p \in X$ and $q \in M$ arbitrary. Since $M$ is connected, there is a path $\gamma: I \rightarrow M$ such that
	$\gamma(0)=p$ and $\gamma(1)=q$. Set \[Y:= \gamma^{-1}(M-X).\] We would like to show that $Y=\emptyset$. Suppose that $Y$ is nonempty and set $y:=\inf(Y)$ and $z=\gamma(y)$. Choose affine coordinates $\varphi: U \rightarrow V$ around $z$ and $ \psi: W \rightarrow Z $ around $f(z)$ such that $f(U) \subseteq W$ and $g(U) \subseteq W$. Since $p$ is in $X$ there exists $l<y$ such that $\gamma(l) \in U\cap X$. Then $f\vert_U$ and $g\vert_U$ coincide in an open neighborhood. Since both $U$ and $ W$ are isomorphic to open subsets of $\R^m$ we conclude that $f\vert_U=g\vert_U$. This is a contradiction.
\end{proof}


\begin{lemma}\label{local}
	Let $M$ be a simply connected complete affine manifold of dimension $m$. Then any affine coordinate $\varphi: U \rightarrow V\subseteq \R^m$ can be extended uniquely to a diffeomorphism from $M$ to $\R^m$.
\end{lemma}
\begin{proof}
	Fix a point $p \in M$ and affine coordinates $\varphi: U \rightarrow V$. By Corollary \ref{extend},
	$\varphi$ can be extended uniquely to an affine immersion $\tilde{\varphi}: M \rightarrow \R^m$. We will prove that $\tilde{\varphi}$ is a diffeomorphism. Since $M$ is complete, the exponential map is defined on the whole tangent space $T_pM$:
	\[ \mathsf{Exp}: T_pM \rightarrow M.\]
	On the other hand, the derivative of $\varphi^{-1}$ at $x=\varphi(p)$ is an isomorphism from $\R^m$ to $T_pM$. We claim that
	the map: \[\psi(y):=\mathsf{Exp} \circ D(\varphi^{-1})(y-x)\]
	is the inverse of $\tilde{\varphi}$. By Lemma \ref{local}, it suffices to show that the functions are inverses to each other in a small open neighborhood. For $q \in U$ we compute:
	\[ \psi (\tilde{\varphi}(q))= \mathsf{Exp} \circ D(\varphi^{-1})(\varphi(q)-x)=q.\]
	Conversely, for $y \in V$ we have:
	\[ \tilde{\varphi}\circ \psi(y)= \varphi \circ \mathsf{Exp} \circ  D(\varphi^{-1})(y-x)= \mathsf{Exp}(y-x)=y. \]
\end{proof}


\begin{proposition}\label{develop} \index{Complete manifold}
	Let $M$ be an affine manifold. The following statements are equivalent:
	\begin{enumerate}
		\item  $M$ is complete.
		\item There is an affine diffeomorphism $M \cong \R^m /\Gamma$, where $\Gamma$ is a discrete subgroup of $\mathsf{Aff}(\R^m)$ and the projection $\pi: \R^m \rightarrow M$ is the universal cover of $M$.
	\end{enumerate}
\end{proposition}

\begin{proof}
	Let us assume that $M$ is complete. Then $\tilde{M}$ is also complete and simply connected. By Proposition \ref{develop} there is an affine difeomorphism $\varphi: \tilde{M} \rightarrow \R^m$. Let us fix a point $p \in M$. Then there is an action of $\pi_1(M,p)$ on $\tilde{M}$ and therefeore on $\R^m$. Since this action is by affine transformations, it defines a homomorphism $\rho: \pi_1(M,p) \rightarrow \mathsf{Aff}(\R^m)$. If we set $\Gamma:= \mathsf{Im}(\rho)$ we get by construction that $ M \cong \R^m/\Gamma$. Conversely, any manifold of the form $\R^m/\Gamma$ is complete.
	
	
\end{proof}